\documentclass{article}
\usepackage[utf8]{inputenc}
\usepackage[margin={.95in,.75in}]{geometry}
\usepackage{titlesec}
\usepackage{titling}
\usepackage[default,scale=1]{opensans}

\setlength{\parindent}{0em}
\setlength{\parskip}{0.5em}
\renewcommand{\baselinestretch}{1.5}
\pagenumbering{gobble}
\setlength{\hyphenpenalty}{1000}
\setlength{\exhyphenpenalty}{1000}

\author{Ryan Bruno}
\date{\today}

\renewcommand{\maketitle}{
\textbf{\underline{Name:}}
\theauthor

\textbf{\underline{Title:}}
Analyzing a Countries Diet with Three Factors
}

\begin{document}

\maketitle

\textbf{\underline{Analysis}}

\setlength{\parindent}{10ex}
Our textbook suggests that the food a country eats is a function of their level development, physical condition, and cultural preference. We can explain how each come together and influence a country's diet starting with the level of development. To explain, we can look at some fundament differences between developed and developing nations. Developing nations have less technology, education, and wealth, therefore, a larger portion of their people farm crops due to the lack of advanced farming techniques or equipment leading to higher food prices. If we compare this to a typical developed country we see a small proportion of the population farm with advanced farm equipment and techniques lowering the cost of food for all people. This leads to developing nations consuming less food then developed due to cost. Along that same line, the cost of transporting food, while getting cheaper, is very expensive. This makes it advantageous to consume foods more easily grown nearby. This explains how the physical conditions of a country influence their diet, however, both level of development and physical condition are not the entire picture; lastly, we must consider culture. Many cultural, both folk and popular, have dietary restrictions and preferences through religion or social norms that have a heavy influence on a countries diet. Now that we know how these three factors influence the food a country eats, we can apply it to the United States. 

The United States is a large and diverse country but we can still use the thee factors to explain America's diet. A normal American consumes a lot of food; about 3,800 kcal per day while the world average is 2,900. Grain constitutes a large amount of farmland but is shared with livestock. This explains America's love for bread, chicken, and beef. Lastly, American culture is very diverse however has some ubiquitous themes. For example, I believe a large consumption of food is within our culture; from grocery stores to restaurants proportions are big and people like it.

Organic food is food that was farmed in a way that meets certain specifications. The goal of these specifications is to help the environment, increase biodiversity, and promote the ecosystem. People in developed countries who believe in such ideals and have the money should opt for organic options. I personally have not made this switch. Money is a big reason but more importantly, I have not done enough research into what these specifications are and the lasting impacts they have as compared to normal farming to make the switch.

The United States and other developed countries have a duty to ensure food security for all people. They do this for their citizens through welfare; in the US we have food stamp programs. However, I believe the only true solution for ensuring food security for both citizens and non-citizens is to eliminate poverty. This is not a trivial problem and we analyzed it last week but to ensure food security for all people you must first eliminate poverty; till then welfare programs are a temporary fix.


\end{document}
