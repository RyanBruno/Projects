\documentclass{article}
\usepackage[utf8]{inputenc}
\usepackage[margin={.95in,.75in}]{geometry}
\usepackage{titlesec}
\usepackage{titling}
\usepackage[default,scale=1]{opensans}

\setlength{\parindent}{0em}
\setlength{\parskip}{0.5em}
\renewcommand{\baselinestretch}{1.5}
\pagenumbering{gobble}

\author{Ryan Bruno}
\date{\today}

\renewcommand{\maketitle}{
\textbf{\underline{Name:}}
\theauthor

\textbf{\underline{Title:}}
An Analysis of the Rise of Religiously Unaffiliated People in The US
}

\begin{document}

\maketitle

\textbf{\underline{Analysis}}

\setlength{\parindent}{10ex}
For most of mankind’s history, religion has played a big role in almost every culture. Very early on humans idealized things such as a river or The Sun. Most modern religions idealize a deity or many deities. Today the level of religious people is the lowest it has ever been and dropping. We can interpret this fact by looking at why this is happening and what are the implications to our community if this continues. With every religion comes a set of beliefs, rules, and values; these traits of religion can help us with our interpretation. 

In early cultures, religions were based around worshiping parts of their environment. For example, a society that lived near a river that worships that river. The river gives them fresh water for crops, drinking, and bathing but often it floods and destroys their farms and houses. Any person today can tell you why the river floods but they do not fully understand that yet. To that early society, it appears the river has a mind of its own, supplying food for the society and taking it at will. This leads to the worship of the river to be merciful to them. However, once the people discovered the reason why the river floods they stopped worshiping the river. This cycle repeats itself until we stopped worshiping things and started worshiping divine beings. My theory as to explain the religious decline in America is that, similar to that early society, current religions are based around not fully understanding our own intelligence, believing it to be a gift from a divine entity, however with new research in neuroscience and artificial intelligence more and more people are rejecting this belief. If this were true it still leaves the question as to what the implications in this decline in religion.

Most religions have beliefs that are associated with rules and morals. These usually are very beneficial to society. For example, if it is against the rules of your religions to murder this adds another deterrent, other then the legal one, to murder. Other religious morals can include donating to charity, helping others, and generally being a good person. For those who are religious and renounce their religion, they do not automatically lose the morals of their religion. On the same note, a parent who is not religious can raise their kid just the same as a religious one. With that being said, religious advocates claim that practicing religion through reading sacred texts, regular worship, and believing in a deity will help you or your child become a better person. If the percent of Americans identify as religiously unaffiliated continues to rise I believe the short term implications would be minimal due to the fact that moral, religious people do not lose their morality when they lose their religion. In the long term, there may be implications as religious morals passed down from generations are lost but I believe the cycle described previously will repeat itself and religion will make a comeback with a reformed or brand new belief system.

\end{document}
