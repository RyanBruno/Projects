\documentclass{article}
\usepackage[utf8]{inputenc}
\usepackage[margin={.95in,.75in}]{geometry}
\usepackage{titlesec}
\usepackage{titling}
\usepackage[default,scale=1]{opensans}

\setlength{\parindent}{0em}
\setlength{\parskip}{0.5em}
\renewcommand{\baselinestretch}{1.5}
\pagenumbering{gobble}

\author{Ryan Bruno}
\date{\today}

\renewcommand{\maketitle}{
\textbf{\underline{Name:}}
\theauthor

\textbf{\underline{Title:}}
An Analysis Poverty in The US
}

\begin{document}

\maketitle

\textbf{\underline{Analysis}}

\setlength{\parindent}{10ex}
A very important approach to analyze a nation is by studying their population of impoverished people. Every nation has poverty; poor nations have a very high rate of poverty but even wealthy nations have a poor population. According to the Census, 13.5\% of US citizens are below the poverty line while 27\% of Hispanic and Black Americans live below the poverty line. The poverty line is set at a yearly income level of \$12,331 for one family member to \$24,257 for four. Poverty is affected by many factors and affects many people in different ways; in this paper, we can analyze it from a few different points of view.

The first thing to look at defines poverty; the poverty line. For one person, \$12,331 a year is not much at all. For reference, most student houses in Shippensburg cost at least \$1,000 per month to live. So making below the poverty line means you cannot afford to live in Shippensburg. In my opinion, this number to low. While people below the line do not make enough to afford basic human needs, I assume there are people above the line who have the same problems.

The second thing to look at is who poverty affects. A higher percentage of Hispanic and Black Americans are in poverty than White Americans. In Hispanic's case, one factor is the recency of immigration. A large population of Hispanics are first-generation or second-generation immigrants. A lot of immigrants start at low wage jobs due to a lack of education and it may take generations until the American Dream can be achieved. This does not apply to Black Americans, who are not recent immigrants. One factor can be explained by looking through history. After the thirteenth amendment in 1865, freed Black slaves in America had to face Jim Crow Laws and other discrimination holding them in poverty. Even after the civil rights movement, it may take generations for Black families to get out of poverty.

The last thing to look at is ways to lower poverty. In the previous paragraph, I mentioned education; which is a great place to start. Areas with higher poverty have less education spending due to having a smaller tax base. This is why I advocate for more federal education spending in those areas. Many factors can hinder a child's education; such as organized crime. This is why I also support programs who focus on keeping children away from gangs and focused on school.

Poverty affects many people; we have talked about those who it affects directly but it affects many more indirectly. I have been fortunate that my parents have never been in poverty but there are those in my family who have. With one uncle who loads boxes for UPS with a bad back to an aunt who sells bath fitters at malls but does not make enough money for the tolls to get there, the hard part from, my point of view, is not being able to do much about it. You can give them money but that is a short-term solution that does little in the long run; poverty is a complex problem that requires a complex solution.


\end{document}
