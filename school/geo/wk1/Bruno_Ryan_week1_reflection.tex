\documentclass{article}
\usepackage[utf8]{inputenc}
\usepackage[margin={.95in,.75in}]{geometry}
\usepackage{titlesec}
\usepackage{titling}
\usepackage[default,scale=1]{opensans}

\setlength{\parindent}{0em}
\setlength{\parskip}{0.5em}
\renewcommand{\baselinestretch}{1.5}
\pagenumbering{gobble}

\author{Ryan Bruno}
\date{\today}

\renewcommand{\maketitle}{
\textbf{\underline{Name:}}
\theauthor

\textbf{\underline{Title:}}
An Analysis of the Push and Pull Factors in My Migration After College.
}

\begin{document}

\maketitle

\textbf{\underline{Analysis}}

\setlength{\parindent}{10ex}
To better understand my analysis I will first describe where I consider my "home" now and in the past. The house I have lived in since I was born is located in Horsham, PA. An hour's drive to Philadelphia; it is located in the suburbs of the city. After that, I have lived in Shippensburg off and on for two and a half years. Compared to Horsham, Shippensburg is more rural. After college, the plan is to get a job and start my new life somewhere. That somewhere is what I am going to discuss in this paper. With that out of the way, I can now describe the push and pull factions for my previous locations and other locations I am considering.

The first location I am considering is the town I grew up in; Horsham. It has plenty of pull factors. The biggest is the economical one. Living in a city as an entry-level employee can mean your rent is half your disposable income. So if I can live with my parents this would allow me to build savings. This along with the familiarity of the town and its people make up most of the pull factors but there are push factors. My desired profession is a software developer and while those jobs exist in and around Philadelphia other cities are better known for those jobs. Shippensburg has similar factors to Horsham. The cost of living is cheap, it's familiar, but there may be better career opportunities elsewhere. Both of these locations I can speak to because of my residency there but I can analyze other locations based on accounts from others, knowledge from my education, and the news.

To begin my analysis of possible locations after college we can rule some out first. I would like to stay in the United State. Besides the complications that arise with visas and citizenships, the differences in culture and distance from family would be too much for me right out of college. This also narrows out the west coast. While I would like to visit some of the national parks out west, it is too far away from family. To narrow down more cities, in general, have many pull factors for me. This includes better job opportunities and social life. If I do have to move away from Horsham and Shippensburg this means I have to meet new people. While this can be done in a more rural area, it would be easier in a city. Then comes the question of which city. I can list push and pulls for each city like; New York and DC have a very high cost of living, Boston is too cold, Houston is too hot, but I cannot make a decision until faced with all options including, job offers and wheather or not I can find a roommate in said city. I am set to graduate May of 2021 and economical pull factors of each city, based on the success of the companies within, can change from now till then. With all this being said, looking at current knowledge, I can decide what my goal locations are.

So for my final answer of where to live after college, given all my current knowledge, is split between my hometown of Horsham, PA or Washington DC. The biggest pull factor in Horsham is that living with my parents will save me a lot of money with a job in Philadelphia. On the flip side, DC is close to my family and is a great place to start a career for me. 

\end{document}
