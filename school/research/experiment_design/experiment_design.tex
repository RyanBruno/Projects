\documentclass{article}
\usepackage[utf8]{inputenc}
\usepackage{multicol}
\usepackage[margin={1.25in,1.25in}]{geometry}
\usepackage{titlesec}
\usepackage{titling}
\usepackage[
    backend=biber,
    style=apa,
    url=false, 
    doi=false,
    eprint=false
]{biblatex}

\usepackage{parskip}

\title{Experiment Design (Draft)}
\author{Ryan Bruno}
\date{\today}

%\addbibresource{bib.bib}

\renewcommand{\maketitle}{
    \begin{center}
        \Huge\textbf{\thetitle}
    \end{center}
    \begin{multicols}{2}
        \begin{flushright}
        By: \theauthor
        \end{flushright}

        \columnbreak
        
        \thedate
    \end{multicols}
}

\begin{document}

\maketitle

\section*{Goal}

To acquire the knowledge of whether or not the use of an Or-Set in a data distributed environment provides the guarantees listed in the objectives when compared to the controlled trial.

\section*{Treatments}

In order to accomplish the goal a single policy (a merging algorithm) must be tested with different variables in the form of different situations of concurrent data modification listed below:

    - No concurrent Modifications

    - Concurrent additions of different items

    - Concurrent additions of the same item

    - Concurrent modification of the same item

    - Concurrent deletion and deletion then addition of the same item

\section*{Trial}

A complete trial would consist of all treatments that include the above listed variables using either the Or-Set merging algorithm or the control.

\section*{Control Trial}

In order to show in benefits of an Or-Set it's trial will be compared with a control trial. The set used in the control trial will keep no meta-data about items but will simply add and remove items from a local set. When nodes merge the resulting set will be a simple set union of both the node's sets.

\section*{Protocol}

Each trial will follow the following steps:

    1. Two nodes are created that are connection via a network connection and initially hold no data.

    2. Both nodes are given the exact same preset, random data.

    3. The two nodes are disconnected from each other such that they cannot share any information.

    4. According to the treatments data is concurrently changed differently on each node.

    5. The nodes are reconnected and may share data then preform the merging algorithm in the current treatment.

    6. The data on each node is inspected to ensure it meets the requirements in the objective.

\section*{Trial Run-Time}

We can estimate the run time of a single trial to be about one person hour.

\end{document}
